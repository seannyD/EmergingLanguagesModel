\documentclass[11pt, oneside]{article}
\usepackage{geometry}
\usepackage{bm}
\usepackage{amssymb}
\usepackage{amsmath}
\usepackage{palatino}
\usepackage[T1]{fontenc}
%\usepackage{newpxtext,newpxmath}
\usepackage{verbatim}
\usepackage{mathtools}
\usepackage{verbatim}
\usepackage{natbib}
\usepackage{placeins}
\usepackage{booktabs}
\usepackage{siunitx}
\usepackage{graphicx}
\usepackage{caption}
\usepackage{subcaption}
\usepackage{changepage}
\linespread{1.3}


\geometry{letterpaper}


\title{Emerging Languages Model}
\author{Sean, Connie, Bill}

\begin{document}
\maketitle

\section{Acquiring \& Using Speech and Signs} % (fold)
\label{sec:Acq}
Indviduals in the model acquire an inventory of signs and speech sounds by observing and using language in minature/one-shot encounters with other individuals.

\subsection{Exemplar Storage} % (fold)
\label{sub:exemplar_storage}
We assume a simple exemplar model of signal storage: whenever a learner observes a signal $x_i$, it stores the signal in a memory $\mathcal{E} = \{x_i\colon i = 1,\ldots,n\}$. Each signal $x_i$ has a number of properties: the modality in which it was expressed, the form of the signal, and the meaning it denotes. We keep track of these properties to summarise an individual's linguistic competance. For example, the balance of signs and speech sounds in an individual's inventory is given by:
\begin{equation}
	b = \frac{1}{n}\sum_{i=1}^n m_i \;\;\;\;\; \mbox{for } i= 1,\ldots,n,
\end{equation}
where $\mathcal{M} = \{m_i\colon i =1,\ldots,n\}$ is a record of the modality for every sign encountered, such that $m_i = 1$ if signal $x_i$ was manual and $m_i = 0$ if $x_i$ was audible. Deaf individuals store only manual signals. Hearing individuals store both manual and audible signals.

\subsection{Modalities in Action} % (fold)
\label{sub:modalities_in_action}
Individuals' use of signals during interactions is governed by a set of simple heuristics. These heuristics express patterns of conditional usage of sign or speech as a function of the interlocutor's deaf status. 

In each `stage', every agent chooses another agent to interact with, in accordance with the social structure weights outlined below (they are more likely to interact with someone close to them in the social structure). The first agent is the speaker and the chosen partner is the listener.  Therefore, every agent transmits a signal, but may receive many or none.

In an interaction, the speaker produces a signal which the listener stores (if possible). If the speaker is a deaf individual, she always produces a manual signal: if she already knows a sign, she selects randomly among these; otherwise, she innovates a new sign. If the speaker is a hearing individual, the modality of the signal produced depends upon the hearing status of the listener. If the listener is a hearing individual, the speaker samples randomly from all exemplars she knows, as such producing a manual sign with probability $b$. If the listener is deaf, she produces a manual sign, either sampled randomly from known signs or innovated if no signs are already known. Table \ref{table:DND} express the probability $P(m_{new} = 1)$ that the the produced signal $m_{new}$ is in the manual modality.

\begin{table}[h]
\centering
\caption{Probabilities that a produced signal is in the manual modality}
\label{table:DND}
\begin{tabular}{clll}
\multicolumn{1}{l}{}     & \multicolumn{1}{c}{\textbf{Deaf Listener}} & \multicolumn{1}{c}{\textbf{Hearing Listener}} &  \\ \cline{1-3}
\textbf{Deaf Speaker}    & $P(m_{new}= 1) =1$                         & $P(m_{new}= 1) =1$                            &  \\
\textbf{Hearing Speaker} & $P(m_{new}= 1) =1$                         & $P(m_{new}= 1) =b$                            &
\end{tabular}
\end{table}
% subsection modalities_in_action (end)
% subsection exemplar_storage (end)
% section definition (end)

A simulation consists of $Param_{nStages}$ stages.

\subsection{Measuring communicative success}

The communicative success of two given agents is measured as the similarity in the distribution of forms in the agents' memories:

\begin{equation}
Comm(agents_i, agents_j) = | b_i - b_j |
\end{equation}

\section{Social structure}

A world contains a fixed number of agents, controlled by $Param_{nAgents}$.  Agents are assigned a random sex (stored in a vector $sex$ and a set of genes (see section below).  Each agent has a hearing status (either hearing or deaf, stored in a vector $deaf$).  

Agents belong to a compound, and a compound belongs to a clan.  Compounds and clans are stored as vectors $clan$ and $compound$ of length $Param_{nAgents}$.  The number of clans is fixed at $Param_{nClans}$.  The maximum number of people in a compound is fixed at $Param_{MaxSizeCompounds}$.

The world is intialised by assigning an even number of agents to each clan (as possible).  Agents are assigned to a compound within the clan.  When the compound is full, a new compound is created.

The social structure $S$ is a matrix of weights (dimensions $Param_{nAgents} \times Param_{nAgents}$) which determines how likely each agent is to interact with every other agent.  $S$ is initialised according to several factors, including whether the agents are in the same compound, the same clan, are of the same sex and are both deaf.  Each factor is weighted by a parameter.  Below, $same$ is a function that returns 1 if the items are the same, and 0 if not.  $both$ is a function that returns 1 if both items are true.

\begin{equation}
\begin{aligned}
S_{i,j} = & same(clan_{i}, clan{j})Param_{clanWeight}   \\
	& + same(compound_i,compound_j)Param_{compoundWeight}  \\
	& + same(sex_i,sex_j)Param_{genderWeight}  \\
	& + both(deaf_i,deaf_j)Param_{deafCommunityWeight})
\end{aligned}
\end{equation}


\section{Marriage}
A matrix $M$ keeps track of which individuals are married.  A matrix $F$ stores the weights of likelihoods that each agent would marry every other agent.  

There are some hard constraints on who can become married ($F_{i,j}$ is set to 0):  Agents are not permitted to have more than $Param_{maxNumberOfMarriagePartners}$ marriage partners.  Agents can only marry an agent with opposing sex.  Beyond those constraints, $F$ is set to reflect

\begin{equation}
\begin{aligned}
F_{i,j} = & same(clan_{i}, clan{j})Param_{clanWeightMarry}   \\
	& + same(compound_i,compound_j)Param_{compoundWeightMarry}  \\
	& + Comm(deaf_i,deaf_j)Param_{comSimilarityMarry}
\end{aligned}
\end{equation}

Where $Comm(i,j)$ is a function that measures the communicative similarity of two agents.

A set of parameters fix the ideal proportion of married agents in the community.  The proportion of married hearing agents is set in $Param_{proportionOfMarriedAgents}$ and the proportion of married deaf agents is set in $Param_{proportionOfMarriedAgentsDeaf}$.

Marriages occur periodically, with agent $i$ and $j$ being selected for marriage with a probability in proportion to $F_{i,j}$.  Couples are selected, then married (updating $M$) until the ideal proportions are satisfied, or no more marriages are permitted.  Marriage is either patrilocal (the female in the couple adopts the male's clan, compound and associated social structure) or matrilocal (the male in the couple adopts the females's clan, compound and associated social structure), depending on $Param_{MarriageLocality}$ (although only patrilocal is implemented, currently).



\section{Death and Reproduction}

Every 3 stages, a number of births and deaths may occur with probability $Param_{probOfBirthDeathEachStage}$.  A number of agents are added to the population, up to $Param_{maxNumberOfBirthDeathEachStage}$.  A random married couple is selected and a child agent is added to the population.  The child inherits its genetics from its parents (see below).  It inherits its clan, compound and social structure weights from either its mother or its father, depending on $Param_{CulturalInheritance}$ (currently only patrilineal is implemented).

After this, an equal number of agents are selected at random to be removed from the population (so the population size remains stable).  

\subsection{Genetics}
Agents carry two genetic alleles, a `deaf' allele and a `hearing' allele.  The deaf allels are recessive, so an agent must have two deaf alleles to possess a deaf phenotype.  Child agents receive one randomly selected allele from each of their 2 parents.

The initial prevelance of genes are controlled by two parameters that control the proportion of recessive genotypes ($Param_{gDom}$) and the proportion of mixed genotypes ($Param_{gCarry}$).

\section{Interpreting stages in years}
Each agent keeps track of the number of stages they have experienced (their `age').  The average age of agents at the end of the simulation is calculated, and equated with the average life expectancy of real people in the given situation.  In the case of Kata Kolok, the average number of stages experienced is roughly 500, and the average life expectancy is roughly 70 years.  That means that 1000 stages is roughly 140 years.

\end{document}

